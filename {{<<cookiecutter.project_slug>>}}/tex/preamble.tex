%==========%
% Preamble %
%==========%

%=== Language ===%
\usepackage[<< cookiecutter.language >>]{babel}

%=== 'geometry' package ===%
\usepackage{geometry}

%=== Colors ===%
\usepackage[svgnames]{xcolor}
\newcommand{\sectioncolor}{\color{purple}}

%=== 'hyperref' package ===%
\usepackage[pdfusetitle]{hyperref}
\hypersetup{
  colorlinks=true,
}

%=== 'graphicx' package ===%
\usepackage{graphicx}

<% if cookiecutter.pythontex %>
%=== 'pythontex' package ===%
\usepackage{pythontex}

<% endif %>
<% if cookiecutter.minted %>
%=== 'minted' package ===%
\usepackage{minted}

%=== 'booktabs' package ===%
\usepackage{booktabs}
\renewcommand{\arraystretch}{1.2}

<% endif %>
%=== Document title ===%
\usepackage{titling}

\pretitle{\begin{center}\huge\sffamily\bfseries}
\posttitle{\end{center}}

\preauthor{\begin{center}\Large\sffamily\begin{tabular}[t]{c}}
\postauthor{\end{tabular}\end{center}}

\predate{\begin{center}\Large\sffamily}
\postdate{\end{center}}

%=== Section titles ===%
\usepackage{titlesec}

<% if cookiecutter.documentclass == "book" or cookiecutter.documentclass == "report" %>
\titleformat{\part}[display]{\Huge\bfseries\filcenter}
  {\sectioncolor{}\scalebox{2.5}{\thepart}}
  {1ex}{}

\titleformat{\chapter}{\LARGE\sffamily\bfseries}
  {\sectioncolor{}\scalebox{2.25}{\thechapter}}
  {1ex}{}

<% endif %>
\titleformat{\section}{\Large\sffamily\bfseries}
  {\sectioncolor{}{\normalfont\sffamily\S}\thesection}
  {1ex}{}

\titleformat{\subsection}{\large\sffamily\bfseries}
  {\sectioncolor{}\thesubsection}
  {1ex}{}

\titleformat{\subsubsection}{\sffamily\bfseries}
  {\sectioncolor{}\thesubsubsection}
  {1ex}{}

%=== Header and footer ===%
\usepackage{fancyhdr}
\pagestyle{fancy}

\setlength{\headheight}{14pt}
\addtolength{\topmargin}{-2pt}

<% if cookiecutter.twoside %>
<% if cookiecutter.documentclass == "book" or cookiecutter.documentclass == "report" %>
\renewcommand{\chaptermark}[1]{%
  \markboth{\thechapter.\ #1}{}}
\renewcommand{\sectionmark}[1]{%
  \markright{\thesection.\ #1}}

% Chapter and section titles in headers
\fancyhf{}
\fancyhead[RE]{\sffamily\bfseries\nouppercase{\leftmark}}
\fancyhead[LO]{\sffamily\bfseries\nouppercase{\rightmark}}
\fancyhead[RO,LE]{\sffamily\bfseries\thepage}
<% else %>
\renewcommand{\sectionmark}[1]{%
  \markright{\thesection.\ #1}}
\renewcommand{\subsectionmark}[1]{%
  \markright{\thesubsection.\ #1}}

% Section and subsection titles in headers
\fancyhf{}
\fancyhead[LE]{\sffamily\bfseries\nouppercase{\rightmark}}
\fancyhead[RO]{\sffamily\bfseries\nouppercase{\leftmark}}
\fancyhead[RE,LO]{\sffamily\bfseries\thepage}
<% endif %>
<% else %>
<% if cookiecutter.documentclass == "book" or cookiecutter.documentclass == "report" %>
\renewcommand{\chaptermark}[1]{%
  \markboth{\thechapter.\ #1}{}}

% Chapter titles in headers
\fancyhf{}
\fancyhead[L]{\sffamily\bfseries\nouppercase{\leftmark}}
\fancyhead[R]{\sffamily\bfseries\thepage}
\fancypagestyle{plain}{
  \fancyhf{}
  \renewcommand{\headrulewidth}{0pt}
  \fancyfoot[C]{\sffamily\thepage}
}
<% else %>
\renewcommand{\sectionmark}[1]{%
  \markboth{\thesection.\ #1}{}}

% Section titles in headers
\fancyhf{}
\fancyhead[L]{\sffamily\bfseries\nouppercase{\leftmark}}
\fancyhead[R]{\sffamily\bfseries\thepage}
<% endif %>
<% endif %>
% Plain pages: no header and centered page number
\fancypagestyle{plain}{
  \fancyhf{}
  \renewcommand{\headrulewidth}{0pt}
  \fancyfoot[C]{\sffamily\thepage}
}

%=== Mathematics ===%
\usepackage{mathtools}
\mathtoolsset{centercolon}

\usepackage{amssymb}

%== Math environments ==%
\usepackage{amsthm}
\usepackage{thmtools}

\declaretheoremstyle[
  headfont=\sffamily\bfseries\color{MediumBlue},
  headpunct={},
  postheadspace=1em,
  spaceabove=1em,
  spacebelow=1em,
]{thmblue}
\declaretheoremstyle[
  headfont=\sffamily\bfseries\color{FireBrick},
  headpunct={:},
  spaceabove=1em,
  spacebelow=1em,
]{thmred}
\declaretheoremstyle[
  headfont=\sffamily\bfseries\color{ForestGreen},
  headpunct={ ---},
  spaceabove=1em,
  spacebelow=1em,
]{thmgreen}
\declaretheoremstyle[
  headfont=\sffamily\bfseries\color{black},
  spaceabove=1em,
  spacebelow=1em,
]{thmblack}

%= Math environments names =%
<% if cookiecutter.language == "spanish" %>
\newcommand{\theoremname}{Teorema}
\newcommand{\propositionname}{Proposición}
\newcommand{\corollaryname}{Corolario}
\newcommand{\lemmaname}{Lema}
\newcommand{\conjecturename}{Conjetura}
\newcommand{\definitionname}{Definición}
\newcommand{\notationname}{Notación}
\newcommand{\examplename}{Ejemplo}
\newcommand{\remarkname}{Nota}
\newcommand{\problemname}{Problema}
\newcommand{\questionname}{Cuestión}
\newcommand{\exercisename}{Ejercicio}
\newcommand{\solutionname}{Solución}
<% elif cookiecutter.language == "catalan" %>
\newcommand{\theoremname}{Teorema}
\newcommand{\propositionname}{Proposició}
\newcommand{\corollaryname}{Corol·lari}
\newcommand{\lemmaname}{Lema}
\newcommand{\conjecturename}{Conjectura}
\newcommand{\definitionname}{Definició}
\newcommand{\notationname}{Notació}
\newcommand{\examplename}{Exemple}
\newcommand{\remarkname}{Nota}
\newcommand{\problemname}{Problema}
\newcommand{\questionname}{Qüestió}
\newcommand{\exercisename}{Exercici}
\newcommand{\solutionname}{Solució}
<% else %>
\newcommand{\theoremname}{Theorem}
\newcommand{\propositionname}{Proposition}
\newcommand{\corollaryname}{Corollary}
\newcommand{\lemmaname}{Lemma}
\newcommand{\conjecturename}{Conjecture}
\newcommand{\definitionname}{Definition}
\newcommand{\notationname}{Notation}
\newcommand{\examplename}{Example}
\newcommand{\remarkname}{Remark}
\newcommand{\problemname}{Problem}
\newcommand{\questionname}{Question}
\newcommand{\exercisename}{Exercise}
\newcommand{\solutionname}{Solution}
<% endif %>

%= Numbered math environments =%
<% if cookiecutter.math_numeration != "plain" %>
\declaretheorem[name=\theoremname, style=thmblue, numberwithin=<<
cookiecutter.math_numeration >>]%
  {theorem}
<% else %>
\declaretheorem[name=\theoremname, style=thmblue]%
  {theorem}
<% endif %>
\declaretheorem[name=\propositionname, style=thmblue, sibling=theorem]%
  {proposition}
\declaretheorem[name=\corollaryname, style=thmblue, sibling=theorem]{corollary}
\declaretheorem[name=\lemmaname, style=thmblue, sibling=theorem]%
  {lemma}
\declaretheorem[name=\conjecturename, style=thmblue, sibling=theorem]%
  {conjecture}
\declaretheorem[name=\definitionname, style=thmred, sibling=theorem]%
  {definition}
\declaretheorem[name=\notationname, style=thmred, sibling=theorem]%
  {notation}
\declaretheorem[name=\examplename, style=thmgreen, sibling=theorem]%
  {example}
\declaretheorem[name=\remarkname, style=thmgreen, sibling=theorem]%
  {remark}

<% if cookiecutter.math_numeration != "plain" %>
\declaretheorem[name=\problemname, style=thmblack, numberwithin=<<
cookiecutter.math_numeration >>]%
  {problem}
<% else %>
\declaretheorem[name=\problemname, style=thmblack]%
  {problem}
<% endif %>
\declaretheorem[name=\questionname, style=thmblack, sibling=problem]%
  {question}
\declaretheorem[name=\exercisename, style=thmblack, sibling=problem]%
  {exercise}

%= Unnumbered math environments =%
\declaretheorem[name=\theoremname, style=thmblue, numbered=no]%
  {theorem*}
\declaretheorem[name=\propositionname, style=thmblue, numbered=no]%
  {proposition*}
\declaretheorem[name=\corollaryname, style=thmblue, numbered=no]%
  {corollary*}
\declaretheorem[name=\lemmaname, style=thmblue, numbered=no]%
  {lemma*}
\declaretheorem[name=\conjecturename, style=thmblue, numbered=no]%
  {conjecture*}
\declaretheorem[name=\definitionname, style=thmred, numbered=no]%
  {definition*}
\declaretheorem[name=\notationname, style=thmred, numbered=no]%
  {notation*}
\declaretheorem[name=\examplename, style=thmgreen, numbered=no]%
  {example*}
\declaretheorem[name=\remarkname, style=thmgreen, numbered=no]%
  {remark*}
\declaretheorem[name=\problemname, style=thmblack, numbered=no]%
  {problem*}
\declaretheorem[name=\questionname, style=thmblack, numbered=no]%
  {question*}
\declaretheorem[name=\exercisename, style=thmblack, numbered=no]%
  {exercise*}

%= Proof and solution environments =%
\newenvironment{coloredproof}[1][MidnightBlue!50!black]%
  {\begin{proof} \color{#1}}%
  {\end{proof}}

\newenvironment{solution}%
  {\begin{proof}[\solutionname] \renewcommand{\qed}{}}%
  {\end{proof}}
\newenvironment{coloredsolution}[1][MidnightBlue!50!black]%
  {\begin{solution} \color{#1}}%
  {\end{solution}}
